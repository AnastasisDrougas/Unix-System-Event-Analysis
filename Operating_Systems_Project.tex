\documentclass[11pt]{article}
\usepackage[a4paper,margin=2.2cm,bottom=2.2cm,footskip=0.9cm]{geometry}
\usepackage{amsmath,amsthm,amssymb,amsfonts, enumitem, fancyhdr, color, comment, environ, mathtools}
\usepackage{graphicx}
\usepackage[greek,english]{babel}
\usepackage{ulem}
\usepackage{float}
\usepackage{caption}
\addto\captionsgreek{\renewcommand{\figurename}{Εικόνα}}
\usepackage{subcaption}
\usepackage{array}        % for \newcolumntype, >{ }, <{ }, \arraybackslash
\usepackage{colortbl}     % for \rowcolor and \rowcolors
\usepackage[table]{xcolor} % for colors (gray!20 etc.)
\usepackage{adjustbox} 
\usepackage{shellesc} % ensures shell-escape works
\usepackage{iftex}    % optional, if you detect engines
\usepackage{ifplatform}
\usepackage{minted}  % for code highlighting
\usepackage{xcolor}
\usepackage[utf8]{inputenc} 

\usepackage{fancyhdr}
\usepackage{xcolor}

\pagestyle{fancy}
\fancyhf{} % Καθαρίζει όλα τα default headers/footers

% Αφαιρούμε την γραμμή πάνω από το header
\renewcommand{\headrulewidth}{0pt}

\renewcommand{\footrule}{%
  \vskip 6pt
  \hrule height \footrulewidth
}


% Στυλ υποσέλιδου
\pagestyle{fancy}
\fancyhf{}

\renewcommand{\headrulewidth}{0pt}
\renewcommand{\footrulewidth}{0.4pt}

\fancyfoot[L]{%
  \raisebox{2ex}{%
    \parbox[t]{\linewidth}{%
      \small\textit{\textcolor{gray}{%
        Δρούγας Αναστάσιος\\
        Λειτουργικά Συστήματα
      }}
    }
  }
}

\fancyfoot[C]{\thepage}

\fancyfoot[R]{%
  \raisebox{-0.2\height}{\includegraphics[height=35pt]{logo_hua.png}}
}
\definecolor{lightgray}{rgb}{0.85,0.85,0.85}

%%%%%%%%%%%%%%%%%%%%%%%%%%%%%%%%%%%%%%
\begin{document}
\selectlanguage{Greek}
%%%%%%%%%%%%%%%%%%%%%%%%%%%%%%%%%%%%%%

\noindent

% Logo στο κέντρο πάνω
\vspace*{-2cm} % ανεβάζει όλο το μπλοκ πιο πάνω

\begin{center}
  \includegraphics[height=40pt]{logo_hua2.png}
\end{center}

% Δύο minipage με τις πληροφορίες
\begin{minipage}[t]{0.45\textwidth}
Δρούγας Αναστάσιος \\
Αριθμός Μητρώου: {\latintext it2024023}
\hrule
\end{minipage}%
\begin{minipage}[t]{0.5\textwidth}
\raggedleft
Λειτουργικά Συστήματα \\
Χειμερινό εξάμηνο 25-26
\hrule
\end{minipage}



\section*{\centering Ανάλυση και Επεξεργασία Συμβάντων Συστήματος\\
σε {\latintext UNIX} Περιβάλλον}

\section*{\normalsize \uline{Εισαγωγή}}


Η εργασία παρουσιάζει την υλοποίηση ενός συστήματος παρακολούθησης συμβάντων σε περιβάλλον {\latintext UNIX}, το οποίο συλλέγει και επεξεργάζεται αρχεία καταγραφής. Μέσω {\latintext shell scripts} και εντολών κελύφους πραγματοποιείται το φιλτράρισμα και η αυτοματοποίηση της ανάλυσης, ενώ προγράμματα σε {\latintext C} χρησιμοποιούνται για την εξαγωγή στατιστικών και τη διαχείριση σφαλμάτων. Τέλος, η χρήση παράλληλης επεξεργασίας με {\latintext threads} επιτρέπει την ταχύτερη ανάλυση πολλαπλών αρχείων {\latintext log}.

\section*{\normalsize \uline{Μέρος Α'}}
\begin{enumerate}
    \item Αρχικά δημιουργούμε τρείς φακέλους με την βοήθεια του {\latintext flag} \textbf{{\latintext -p}} που επιτρέπει την ιεραρχία φακέλων με πολλά επίπεδα:
    {\latintext monitor/{raw,processed,reports}}:

    \begin{center}
    \includegraphics[width=0.9\textwidth]{fileMaking.png}
    \captionof{figure}{\footnotesize Δημιουργία φακέλων με {\latintext mkdir}.}
    \label{fig:filemaking}
    \end{center}

    \item Έπειτα δημιουργούμε τα τρία αρχεία καταγραφής:
    \textbf{{\latintext system.log, network.log, security.log}}:

    \begin{center}
    \includegraphics[width=0.9\textwidth]{LogfileMaking.png}
    \captionof{figure}{\footnotesize Δημιουργία αρχείων με {\latintext touch}.}
    \label{fig:logfilemaking}
    \end{center}

  \item Στο πλαίσιο της παρακολούθησης συμβάντων, δημιουργούνται ενδεικτικές γραμμές καταγραφής με τη χρήση της εντολής {\textbf{\latintext echo}} , σε συνδυασμό με τον τελεστή ανακατεύθυνσης, ο οποίος προσθέτει το περιεχόμενο στο τέλος του αρχείου {\latintext log} χωρίς να διαγράφει τα ήδη υπάρχοντα δεδομένα.

\begin{itemize}
    \item Γραμμές που ξεκινούν με ημερομηνία ({\latintext YYYY-MM-DD}).
    \item Γραμμές που περιέχουν {\latintext ERROR}, {\latintext FAILED}, {\latintext CRITICAL}.
    \item Γραμμές που περιέχουν {\latintext IPv4} διευθύνσεις.
\end{itemize}

\begin{center}
    \includegraphics[width=0.9\textwidth]{echoLogs.png}
    \captionof{figure}{\footnotesize Ενδεικτικές γραμμές καταγραφής.}
    \label{fig:echoLogs}
    \end{center}
    
\item Με την εντολή {\latintext{\textbf{ls -l raw/}}} εμφανίζουμε αναλυτικά όλα τα αρχεία καταγραφής που βρίσκονται στο φάκελο {\latintext{raw/}}, μαζί με πληροφορίες όπως μέγεθος, δικαιώματα και ημερομηνία τελευταίας τροποποίησης.  
Ο συνολικός αριθμός των γραμμών όλων των αρχείων {\latintext log} μπορεί να υπολογιστεί με την εντολή {\latintext{\textbf{wc -l raw/*.log}}}, η οποία μετράει τις γραμμές κάθε αρχείου και δίνει το άθροισμα.


\begin{center}
    \includegraphics[width=0.7\textwidth]{merosA4.png}
    \captionof{figure}{\footnotesize Πληροφορίες αρχείων.}
    \label{fig:merosA4}
    \end{center}
\end{enumerate}







\section*{\normalsize \uline{Μέρος Β'}}
\begin{enumerate}
    \item Χρησιμοποιώντας {\textbf{\latintext grep regex}} μπορούμε να εξάγουμε δεδομένα από τα αρχεία καταγραφείς:
    
\begin{itemize}

  \item Γραμμές που ξεκινούν με ημερομηνία ({\latintext YYYY-MM-DD}).

\begin{otherlanguage}{english}
\begin{minted}[fontsize=\small, linenos, bgcolor=lightgray]{bash}
grep -E '^[0-9]{4}-[0-9]{2}-[0-9]{2}'
\end{minted}
\end{otherlanguage}

\item Γραμμές που περιέχουν {\latintext ERROR}, {\latintext FAILED}, {\latintext CRITICAL}.

\begin{otherlanguage}{english}
\begin{minted}[fontsize=\small, linenos, bgcolor=lightgray]{bash}
grep -E 'ERROR|FAILED|CRITICAL'
\end{minted}
\end{otherlanguage}

\item Γραμμές που περιέχουν {\latintext IPv4} διευθύνσεις.

\begin{otherlanguage}{english}
\begin{minted}[fontsize=\small, linenos, bgcolor=lightgray]{bash}
grep -E '([0-9]{1,3}\.){3}[0-9]{1,3}'
\end{minted}
\end{otherlanguage}

\end{itemize}
Συνδιάζουμε αυτά σε μία εντόλη με χρήση λογικού τελεστή Ή, και τα αποθηκεύουμε στο αρχείο \textbf{{\latintext processed/alerts.raw}}

\begin{center}
    \includegraphics[width=0.9\textwidth]{grepRegex.png}
    \captionof{figure}{\footnotesize  {\latintext grep} με {\latintext regular expressions (regex)}.}
    \label{fig:grepRegex}
\end{center}
    
   \item Στη συνέχεια, αφαιρούμε τις διπλότυπες γραμμές και ταξινομούμε τα δεδομένα, δημιουργώντας το αρχείο {\latintext{\textbf{processed/alerts.sorted}}} με τη χρήση της εντολής {\latintext{\textbf{sort -u}}}.


\begin{center}
    \includegraphics[width=0.9\textwidth]{sort.png}
    \captionof{figure}{\footnotesize {\latintext sorting}}.
    \label{fig:grepRegex}
\end{center}

'Ετσι, ελέγχουμε τα περιεχόμενα των δύο αρχείων (\textbf{{\latintext alerts.raw, alerts.sorted}}) με την εντολή \textbf{{\latintext cat filename}}.

 \begin{figure}[ht]
    \centering
    % Πρώτη εικόνα
    \begin{subfigure}[b]{0.45\textwidth}
        \centering
        \includegraphics[width=\textwidth]{alerstRaw2.png}
        \caption{\latintext alerts.raw}
        \label{fig:img1}
    \end{subfigure}
    \hfill
    % Δεύτερη εικόνα
    \begin{subfigure}[b]{0.45\textwidth}
        \centering
        \includegraphics[width=\textwidth]{alertsSorted.png}
        \caption{\latintext alerts.sorted}
        \label{fig:img2}
    \end{subfigure}
    \caption{Πριν και μετά το \latintext sort.}
    \label{fig:two_images}
\end{figure}   
\end{enumerate}







\section*{\normalsize \uline{Μέρος Γ'}}

\begin{itemize}
\item Αρχικά μετράμε με {\textbf{\latintext wc -l}} όλες τις γραμμές δηλαδή τα {\latintext total alerts}, απο το αρχείο {\textbf{\latintext processed\_alerts.sorted}} με {\latintext input redirection}, το οποίο διαβάζει περιεχόμενο από ένα αρχείο αντί να το πάρει η εντολή από το {\latintext stdin}.

\begin{otherlanguage}{english}
\begin{minted}[fontsize=\small, linenos, bgcolor=lightgray]{bash}
wc -l < processed/alerts.sorted
\end{minted}
\end{otherlanguage}

\item Μέ {\textbf{\latintext grep -c}} μετράμε πόσες γραμμές στο αρχείο ταιριάζουν με το {\latintext pattern}. Εδώ δεν χρειάζεται {\latintext input redirection} για το διάβασμα από το αρχείο {\textbf{\latintext processed\_alerts.sorted}}.

\begin{otherlanguage}{english}
\begin{minted}[fontsize=\small, linenos, bgcolor=lightgray]{bash}
grep -c 'ERROR' processed/alerts.sorted
\end{minted}
\end{otherlanguage}

\begin{otherlanguage}{english}
\begin{minted}[fontsize=\small, linenos, bgcolor=lightgray]{bash}
grep -c '^192\.168\.' processed/alerts.sorted
\end{minted}
\end{otherlanguage}
\end{itemize}

Χρησιμοποιώντας τις παραπάνω εντολές σαν μία παραπάνω εντολή με {\textbf{{\latintext echo} \texttt{\$(...)}}
το οποίο επιτρέπει την εκτέλεση εντολών μέσα στην ίδια γραμμή και την αντικατάσταση τους με το αποτέλεσμα τους, ενώ το \texttt{\>} γράφει το αποτέλεσμα στο αρχείο \textbf{{\latintext reports/daily\_summary.txt}}, δημιουργώντας έτσι μια σύντομη αναφορά από τα {\latintext logs}.\\
Καταφέραμε να εξάγουμε:
\begin{itemize}
    \item  Πόσα {\latintext total alerts} υπάρχουν.
    \item  Πόσα {\latintext ERRORS} υπάρχουν.
    \item Πόσα αφορούν το {\latintext IP range} 192.168.*
\end{itemize}

\begin{center}
    \includegraphics[width=0.9\textwidth]{merosG.png}
    \captionof{figure}{\footnotesize {\latintext Data extraction.}}.
    \label{fig:grepRegex}
\end{center}

Έτσι παράγεται ενα αρχείο \textbf{{\latintext reports/daily\_summary.txt}} του οποίου τα περιεχόμενα εχουν την δομή:
\textbf{\small{\latintext{TOTAL ALERTS: X, ERRORS: Y, LOCAL\_NETWORK\_EVENTS: Z}.}}






\section*{\normalsize \uline{Μέρος Δ'}}
\begin{enumerate}
    \item Αρχικά, δημιουργούμε ενα {\latintext shell script} το \textbf{{\latintext timestamp\_loop.sh}} το οποίο περιέχει μια πολύ απλή {\latintext while} επανάληψη. Θέλουμε η διέργασία αυτή τρέχει στο {\latintext background} και να γράφει περιοδικά
{\latintext timestamps} στο αρχείο \textbf{{\latintext monitor/raw/timestamps.log}}.\\
 Ακολουθεί ο κώδικας του \latintext script.

 \begin{otherlanguage}{english}
\begin{minted}[fontsize=\small, linenos, bgcolor=lightgray]{bash}
#!/bin/bash
while true; do
    echo "$(date) TIMESTAMP_LOOP" >> monitor/raw/timestamps.log
    sleep 7
done
\end{minted}
\end{otherlanguage}

'Ετσι, κάνουμε {\latintext compile} αυτο το {\latintext script} με \textbf{{\latintext chmod +x}}, και το εκτελούμε ακολουθώντας το με \texttt{\&} ώστε η διέργασία αυτή να τρέχει στο {\latintext background} και να γράφει περιοδικά
{\latintext timestamps} στο αρχείο.

\begin{center}
    \includegraphics[width=0.9\textwidth]{runLoopScript.png}
    \captionof{figure}{\footnotesize {\latintext Running timestamp\_loop.sh script}.}
    \label{fig:grepRegex}
\end{center}

\item 'Επειτα, όπως βλέπουμε απο την παραπάνω εικονα, με \textbf{{\latintext ps}} και \textbf{{\latintext grep}} εντοπίζουμε το {\latintext process ID (PID)} της διεργασίας. Το {\latintext aux} μας επιτρέπει να πάρουμε την πλήρη λίστα όλων των {\latintext processes} και το {\latintext [t]} δεν ταιριάζει το ίδιο το {\latintext grep command}. 'Ετσι έχουμε {\latintext PID = 4990.} 

\item Θέλουμε να αλλάξουμε την προτεραιότητα της διεργασίας. Η εντολή \textbf{{\latintext renice}} χρησιμοποιήθηκε για τη μείωση της προτεραιότητας της διεργασίας που ήδη τρέχει ({\latintext nice} από 0 σε 10), ώστε να εκτελείται στο παρασκήνιο χωρίς να επιβαρύνει τις υπόλοιπες διεργασίες του συστήματος.
\begin{center}
    \includegraphics[width=0.9\textwidth]{renice.png}
    \captionof{figure}{\footnotesize Αλλαγή προτεραιότητας διεργασίας.}
    \label{fig:grepRegex}
\end{center}

\item Τέρματιζούμε την διεργασία που τρέχει στο παρασκήνιο με \textbf{{\latintext kill -TERM PID}}, όπως βλέπουμε δεν χρείαστηκε {\latintext KILL}.

\begin{center}
    \includegraphics[width=0.9\textwidth]{Terminate.png}
    \captionof{figure}{\footnotesize {\latintext Terminating backround process.}}
    \label{fig:grepRegex}
\end{center}

\item Τέλος, καταγράφουμε στο αρχείο \textbf{{\latintext process\_log.txt}} όλες τις ενέργειες που κάναμε για το συγκεκριμένο {\latintext PID}.

\begin{center}
    \includegraphics[width=0.9\textwidth]{final.png}
    \captionof{figure}{\footnotesize Αποθήκευση κινήσεων σε {\latintext .txt file.}}
    \label{fig:grepRegex}
\end{center}
\end{enumerate}







\section*{\normalsize \uline{Μέρος E'}}
 Υλοποιούμε το πρόγραμμα \textbf{{\latintext analyze.c}} για την ανάλυση αρχείων καταγραφής. Λαμβάνει από τη γραμμή εντολών το όνομα ενός αρχείου και επιχειρεί να το ανοίξει χρησιμοποιώντας τη συνάρτηση \textbf{{\latintext open()}}. Σε περίπτωση αποτυχίας, αξιοποιεί το \textbf{{\latintext errno}} και τη \textbf{{\latintext perror()}} για την εμφάνιση κατάλληλου μηνύματος σφάλματος και τερματίζει.
 Το πρόγραμμα επιστρέφει:
 \begin{itemize}
     \item 0: επιτυχία
     \item 1: σφάλμα ανοίγματος
     \item 2: κενό αρχείο 
 \end{itemize}

 Ο {\latintext source} κώδικας του προγράμματος βρίσκεται τόσο στον φάκελο της εργασίας όσο και στο {\latintext GitHub repository}, από όπου μπορείτε να τον κάνετε {\latintext clone}:
 \textcolor{blue}{{\latintext \href{https://github.com/AnastasisDrougas/Unix-System-Event-Analysis}}}.\\

 Στο πρόγραμμα υπάρχουν τρεις βασικές συναρτήσεις. Πιο αναλυτικές περιγράφες του κώδικα υπάρχουν ήδη στα σχόλια.
 
 \begin{enumerate}
     \item Η συνάρτηση \textbf{{\latintext file\_checking}} ελέγχει πρώτα ότι ο χρήστης έχει δώσει ακριβώς ένα όρισμα, το οποίο αναμένεται να είναι το αρχείο {\latintext log}. Αν όχι, εμφανίζει μήνυμα  σωστής χρήσης και τερματίζει. Στη συνέχεια, χρησιμοποιεί συναρτήσεις από τη βιβλιοθήκη {\latintext string.h} ({\textbf{{\latintext strlen, strcmp}}) για να ελέγξει ότι το όνομα του αρχείου τελειώνει σε {\latintext .log}. Αν ο έλεγχος αποτύχει, ενημερώνει τον χρήστη και τερματίζει το πρόγραμμα.

     \begin{otherlanguage}{english}
\begin{minted}[fontsize=\small, linenos, bgcolor=lightgray]{c}
int print_and_return_retval(int total_lines, int total_errors, int total_digits);
\end{minted}
\end{otherlanguage}

\item Η συνάρτηση \textbf{{\latintext extract\_data}} διαβάζει το αρχείο χαρακτήρα-χαρακτήρα χρησιμοποιώντας την \textbf{{\latintext read()}}, συγκεντρώνει κάθε γραμμή σε έναν {\latintext buffer} και μετρά:
\begin{itemize}
    \item το συνολικό αριθμό γραμμών.
    \item τις γραμμές που περιέχουν τη λέξη {\latintext ERROR} (με {\textbf{{\latintext strstr}}).
    \item τις γραμμές που περιέχουν ψηφία (με {\textbf{{\latintext isdigit}}).
    
\end{itemize}

Η διαδικασία επαναλαμβάνεται για κάθε μπλοκ δεδομένων που διαβάζεται και περιλαμβάνει ειδική διαχείριση της τελευταίας γραμμής αν το αρχείο δεν τελειώνει με \textbf{\latintext \textbackslash n}
. Στο τέλος, η συνάρτηση επιστρέφει τα αποτελέσματα μέσω της \textbf{{\latintext print\_and\_return\_retval}}.

\begin{otherlanguage}{english}
\begin{minted}[fontsize=\small, linenos, bgcolor=lightgray]{c}
int extract_data(int fd);
\end{minted}
\end{otherlanguage}

\item H συνάρτηση \textbf{{\latintext print\_and\_return\_retval}} παρέχει αναφορά αποτελεσμάτων και καθορίζει τον κωδικό εξόδου του προγράμματος ανάλογα με την κατάσταση του αρχείου.

\begin{otherlanguage}{english}
\begin{minted}[fontsize=\small, linenos, bgcolor=lightgray]{c}
int print_and_return_retval(int total_lines, int total_errors, int total_digits);
\end{minted}
\end{otherlanguage}

\end{enumerate}

Παρακάτω κάποια ενδεικτικά {\latintext runs} του προγράμματος \textbf{{\latintext analyze.c}}.

\begin{figure}[h]
    \centering
    \begin{minipage}{0.45\textwidth}
        \centering
        \includegraphics[width=\linewidth]{SYSTEM2.png}
        \caption*{\footnotesize {{\latintext Running system.log}.}}
    \end{minipage}
    \begin{minipage}{0.45\textwidth}
        \centering
        \includegraphics[width=\linewidth]{SECURITY2.png}
        \caption*{\footnotesize {{\latintext Running security.log}.}}
    \end{minipage}
    \begin{minipage}{0.45\textwidth}
        \centering
        \includegraphics[width=\linewidth]{TIMESTAMPS2.png}
        \caption*{\footnotesize {{\latintext Running timestamps.log}.}}
    \end{minipage}
    \begin{minipage}{0.45\textwidth}
        \centering
        \includegraphics[width=\linewidth]{TEST2.png}
        \caption*{\footnotesize {{\latintext Running a file that is not .log}.}}
    \end{minipage}
\end{figure}








\section*{\normalsize \uline{Μέρος ΣΤ'}}


Καλούμαστε να υλοποιήσουμε ενα \latintext shell script},το \textbf{{\latintext run\_monitor.sh}} , που θα αυτοματοποιεί την ανάλυση αρχείων από έναν φάκελο. Ελέγχει την εγκυρότητα του καταλόγου, κατηγοριοποιεί τα {\latintext log files} ανάλογα με το όνομά τους και καλεί το εκτελέσιμο \textbf{{\latintext analyse\_log}} (από το πρόγραμμα που υλοποίσαμε προηγουμένως). Τα αποτελέσματα συγκεντρώνονται σε ένα ενιαίο αρχείο αναφοράς, το \textbf{{\latintext ./monitor/reports/full\_reports.txt}}.Ο {\latintext source} κώδικας του {\latintext σ shell script} βρίσκεται επίσης στο {\latintext GitHub repo}.
\\
Παρακάτω, σημεία από τον ο κώδικα του {\latintext shell script}:
\\
\\
\\
\begin{itemize}
    \item Έλεγχος για εγκυρότητα φακέλου με \textbf{{\latintext -d "dir"}}:
    
\begin{otherlanguage}{english}
\begin{minted}[fontsize=\small, linenos, bgcolor=lightgray]{c}
if [ ! -d "$dir" ];then
    echo "Directory '$dir' does not exist!"
    exit 1
fi
\end{minted}
\end{otherlanguage}
    
\item Έλεγχος αν ο φάκελος είναι άδειος με \textbf{{\latintext -z "(dir data)"}}, δηλαδή αν τα περιεχόμενα ειναι μήδεν:

\begin{otherlanguage}{english}
\begin{minted}[fontsize=\small, linenos, bgcolor=lightgray]{bash}
#Checking if the given directory is empty.
if [ -z "$(ls -A "$dir")" ];then
    echo "Directory '$dir' is Empty!"
    exit 0
fi
\end{minted}
\end{otherlanguage}

    \item Έλεγχος αν το εκτελέσιμο αρχείο είναι εκτελέσιμο με \textbf{{\latintext -x "(exec\_file)"}}:

    \begin{otherlanguage}{english}
\begin{minted}[fontsize=\small, linenos, bgcolor=lightgray]{bash}
#Checking if the analyse_log is executable
if [ ! -x ./analyse_log ];then
    echo "Error: ./analyse_log is not executable!"
    exit 1
fi
\end{minted}
\end{otherlanguage}

    \item Τέλος, αν όλοι οι έλεγχοι ήταν έγκυροι, για κάθε αρχείο στον φάκελο μας ξεχωριστά εκτελούμε το \textbf{{\latintext ./analyse\_log}}, με την βοήθεια της επανάληψης \textbf{{\latintext for}}. Ταυτόχρονα, κατηγοριοποιούμε τα περιεχόμενα (με \textbf{{\latintext case}}) και τα αποθηκεύουμε στο αρχείο \textbf{{\latintext ./monitor/reports/full\_reports.txt}}.

   \begin{otherlanguage}{english}
\begin{minted}[fontsize=\small, linenos, bgcolor=lightgray]{bash}
#For each file in this directory:
for file in "$dir"/*;do
    if [ -f "$file" ]; then
        echo "Processing $file.."

        case "$(basename "$file")" in
            network.log)
                echo "NETWORK: '$file' ->" >> "$report"
                ;;
            system.log)
                echo "SYSTEM: '$file' ->" >> "$report"
                ;;
            security.log)
                echo "SECURITY: '$file' ->" >> "$report"
                ;;
            *)
                echo "OTHER: '$file' ->" >> "$report"
                ;;
        esac
        ./analyse_log "$file" >> "$report"
    fi 
done
\end{minted}
\end{otherlanguage}
\end{itemize}


Παρακάτω μερικά ενδεικτικά {\latintext runs} του \textbf{{\latintext run\_monitor.sh}}:

\begin{figure}[H]
    \centering

    \begin{subfigure}{0.7\textwidth}
        \centering
        \includegraphics[width=\textwidth]{SHELLRUN.png}
        \caption{\latintext Running the script.}
        \label{fig:shellrun}
    \end{subfigure}

    \vspace{0.5cm}

    \begin{subfigure}{0.7\textwidth}
        \centering
        \includegraphics[width=\textwidth]{TXTFILEDATA.png}
        \caption{\latintext full\_reports.txt data.}
        \label{fig:txtdata}
    \end{subfigure}

    \caption{Εκτέλεση και αποτελέσματα του \latintext shell script.}
    \label{fig:run-monitor}
\end{figure}






\section*{\normalsize \uline{Μέρος Ζ'}}

Τέλος, πρέπει να παραλληλοποιήσουμε το σύστημά μας. Αυτό θα γίνει χρησιμοποιώντας {\latintext threads}. 'Ετσι, στο πρόγραμμα \textbf{\latintext parallel\_analyse.c} για κάθε αρχείο δημιουργείται ξεχωριστό νήμα που μετρά τον αριθμό γραμμών και τις γραμμές που περιέχουν τη λέξη {\latintext ERROR}. Στο τέλος εμφανίζονται τα αποτελέσματα ανά αρχείο καθώς και τα συνολικά στατιστικά. Ο {\latintext source} κώδικας βρίσκεται και αυτός στο {\latintext GitHub repo}.
\\
\\
\\
Έχουμε παρόμοιες συναρτήσεις με το πρόγραμμα {\textbf{{\latintext analyse.c}}, με μερικές παραμετροποιήσεις. Πιο αναλυτική περιγραφή στα σχόλια του κώδικα.

\begin{enumerate}
    \item Αρχικά, δημιουργούμε τα παρακάτω {\latintext structs}:
    \begin{itemize}
    
        \item Δομή 1: Για να δίνεται σε κάθε {\latintext thread} το δικό του {\latintext filename} με ασφάλεια και για να αποφύγουμε τη χρήση {\latintext global} μεταβλητής.

\begin{otherlanguage}{english}
\begin{minted}[fontsize=\small, linenos, bgcolor=lightgray]{c}
//Struct given to each thread
typedef struct {
    char filename[256];
}thread_arg_t;
\end{minted}
\end{otherlanguage}

        \item  Δομή 2: Αποθηκεύει τα αποτελέσματα ενός {\latintext thread} για να τα επιστρέψει στη {\latintext main}.

\begin{otherlanguage}{english}
\begin{minted}[fontsize=\small, linenos, bgcolor=lightgray]{c}
//Struct extracted from each thread
typedef struct{
    char filename[256];
    int total_lines;
    int total_errors;
    int total_digits;
}results_t;
\end{minted}
\end{otherlanguage}
\end{itemize}

    \item H συνάρτηση \textbf{{\latintext file\_checking}}, είναι παρόμοια με αυτή του προγράμματος \textbf{{\latintext analyse.c}}. Μία από τις αλλαγές είναι ότι επιτρέπει την εισαγωγή περισσότερων από ένα αρχεία. Επίσης με μια επαναληπτική διασικασία ελέγχει αν το κάθε αρχείο που δόθηκε, έχει κατάληξη {\latintext .log}.

\begin{otherlanguage}{english}
\begin{minted}[fontsize=\small, linenos, bgcolor=lightgray]{c}
void file_checking(int argc, char*argv[]);
\end{minted}
\end{otherlanguage}

    \item Η συνάρτηση \textbf{{\latintext file\_thread\_handling}}, είναι η συνάρτηση που δίνεται σαν όρισμα σε κάθε {\latintext thread}. Το κάθε {\latintext thread} ανοίγει το αρχείο, μετρά γραμμές και {\latintext errors} μέσω της \textbf{{\latintext extract\_data}}, και επιστρέφει τα αποτελέσματα στη {\latintext main}.

\begin{otherlanguage}{english}
\begin{minted}[fontsize=\small, linenos, bgcolor=lightgray]{c}
void* file_thread_handling(void*arg);
\end{minted}
\end{otherlanguage}   

    \item H συνάρτηση \textbf{{\latintext extract\_data}}, είναι επίσης παρόμοια με αυτή του προγράμματος \textbf{{\latintext analyse.c}}. Αυτό που άλλαξε είναι ότι, αφαιρέθηκε ο υπολογισμως των γραμμών που περιέχουν αριθμητικά ψηφία.

\begin{otherlanguage}{english}
\begin{minted}[fontsize=\small, linenos, bgcolor=lightgray]{c}
void extract_data(int fd, results_t *res, char *filename);;
\end{minted}
\end{otherlanguage}

     \item H συνάρτηση \textbf{{\latintext print\_data}}, είναι η συνάρτηση εκτύπωσης. Τα αποτελέσματα εμφανίζονται με την εξής δομή:

\begin{center}
    \includegraphics[width=0.5\textwidth]{printStructure.png}
    \captionof{figure}{\footnotesize Δομή εκτύπωσης.}
    \label{fig:grepRegex}
\end{center}

\begin{otherlanguage}{english}
\begin{minted}[fontsize=\small, linenos, bgcolor=lightgray]{c}
void print_data(int file_number, pthread_t threads[]);
\end{minted}
\end{otherlanguage}


    \item Τέλος η \textbf{{\latintext main}}, δημιουργεί ένα {\latintext thread} για κάθε αρχείο περνώντας του το όνομα, περιμένει με \textbf{{\latintext pthread\_join}} να ολοκληρωθούν όλα τα {\latintext threads} και στη συνέχεια συγκεντρώνει και τυπώνει τα αποτελέσματα  με την χρήση της \textbf{{\latintext print\_data}}. 
\end{enumerate}
Κώδικας \textbf{{\latintext main}}:


\begin{otherlanguage}{english}
\begin{figure}[H]
\centering
\begin{minted}[fontsize=\small, linenos, bgcolor=lightgray]{c}
int main(int argc, char *argv[]){
    //Check the file structure.
    file_checking(argc, argv);
    //Handle multiple files with threads:
    int file_number = argc-1;
    pthread_t threads[file_number];
    thread_arg_t thread_args[file_number];
    //Create Threads
    for(int i = 0; i < file_number; i++){
        //Extract filename.
        strcpy(thread_args[i].filename, argv[i+1]);
        //Create.
        pthread_create(&threads[i], NULL, file_thread_handling, &thread_args[i]);
    }
    //Print all data extracted.
    print_data(file_number, threads);
    return 0;
}
\end{minted}
\end{figure}
\end{otherlanguage}

Ενδεικτικά {\latintext runs} του \textbf{\latintext parallel\_analyse.c}:

\begin{figure}[H]
    \centering
    \begin{subfigure}{0.7\textwidth}
        \centering
        \includegraphics[width=\textwidth]{PARRUN11.png}
        \caption{\latintext Running with .log files.}
        \label{fig:shellrun}
    \end{subfigure}
    \vspace{0.5cm}
    \begin{subfigure}{0.7\textwidth}
        \centering
        \includegraphics[width=\textwidth]{PARRUN22.png}
        \caption{\latintext Running with empty .log file.}
        \label{fig:txtdata}
    \end{subfigure}

    \caption{Εκτέλεση του \textbf{\latintext parallel\_analyse.c}}.
    \label{fig:run-monitor}
\end{figure}

Συμπερασματικά, ο πηγαίος κώδικας όλων των {\latintext C} προγραμμάτων, των {\latintext shell scripts}, καθώς και ένα {\latintext README.md} με αναλυτικές οδηγίες χρήσης, περιλαμβάνονται στον φάκελο της εργασίας και είναι διαθέσιμα στο ακόλουθο {\latintext GitHub repository}:
\textcolor{blue}{{\latintext \href{https://github.com/AnastasisDrougas/Unix-System-Event-Analysis}{https://github.com/AnastasisDrougas/Unix-System-Event-Analysis}}}
\end{document}
